\documentclass[14pt]{extarticle}
\usepackage{extsizes}
\usepackage{geometry}
\geometry{margin=0.5in}

%% for images
\usepackage{graphicx}
\graphicspath{ {images/} }

%% language support
\usepackage[T1,T2A]{fontenc}
\usepackage[utf8]{inputenc}
\usepackage[english,russian]{babel}

\usepackage{amsmath}
\usepackage{tikz}

%% hyperrefs
\usepackage{hyperref}
\hypersetup{
    colorlinks,
    citecolor=black,
    filecolor=black,
    linkcolor=black,
    urlcolor=black
}

\title{2024}
\author{КЫК и его друзья}

\begin{document}
\maketitle
\tableofcontents

\part*{Глоссарий}

КС - контекстно-свободная

\part{КЯК парсер}
*первый абзац скипнем*
\\\\
Как и с парсером Унгера, на вход алгоритма КЯК подается
КС грамматика и входная последовательность.
Первый этап алгоритма создаёт таблицу, где написано, какие 
нетерминалы производят какие подстроки предложения. 
Это этап распознавания; в итоге он также говорит нам, может
ли входная последовательность быть произведена из грамматики.
Второй этап использует эту таблицу распознавания и грамматику,
чтобы создать все возможные производные предложения. 
Мы сконцентрируемся сначала на этапе распознавания, который
является отдельной особенностью алгоритма. 

\section{КЯК распознавание с обобщёнными КС грамматиками.}

Чтобы увидеть, как алгоритм КЯК решает проблему распознавания
и парсинга, посмотрим на грамматику на рисунке 4.6. 
Эта грамматика описывает синтаксис номеров в экспоненциальной
форме записи. \\
\\
Число $\to$ Целое | Вещественное 
\\
Целое $\to$ Цифра | Целое Цифра 
\\
Вещественное $\to$ Целое Дробь Степень 
\\
Дробь $\to$ . Целое
\\
Степень $\to$ е Знак Целое | Пусто
\\
Цифра $\to$ 0 | 1 | 2 | 3 | 4 | 5 | 6 | 7 | 8 | 9
\\
Знак $\to$ + | -
\\
Пусто $\to \epsilon$\\\\
\textbf{Рис. 4.6.} Грамматика, описывающая числа в экспоненциальной форме записи.
\\\\
Пример предложения, производимого этой грамматикой: 
32.5е+1. Мы будем использовать эту грамматику и предложение
в качестве примера. 

\end{document}

